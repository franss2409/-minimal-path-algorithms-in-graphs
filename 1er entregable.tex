\documentclass[a4,papper]{IEEEtran}
\usepackage[utf8]{inputenc}
\usepackage[spanish]{babel}
\title{Algoritmos de corte mínimo en grafos}
\author{Franss Cruz,\thanks{fransscruz18@gmail.com}
Davis García\thanks{yums123@hotmail.com},Fernando Macuri,\thanks{rfmo2014@hotmail.com}}
%%%%%%%%%%%%%%%%%%%%%%%%%%%%%%%%%%%%%%%%%%%%%%%%%%%%%%%%%%%%%%%%%%%%%%%%%%%%%%%%%%%%%%%%%%%%%%%%%%%%%%%%%%%%%
\begin{document}
\maketitle
\section{Introducción}
\IEEEPARstart{L}a tería de probabilidad proporciona varios modelos matemáticos para la descripción de fenómenos sujeto a influjos csuales, y tiene como objetivo escemcial la compresión matemática de las regularidades de los fenomenos aleatorios.\\
La teoría de probabilidad se construye de manera axiomática, de acuerdo con procedimento probado y muy utilizado hasta hoy día, y se sirve en gran medida de los métodos y resultados del análisis.\\
La Estadística matemática proporciona, sobre la base de la teoría de probabilidades, métodos mediante los cuales se puede obtener información sobre distintas poblaciones e investigar, utilizando datosmuestrales aleatorios;con esto se da origen también a métodos de ajuste de un modelo matemático, que considere efectos aleatorios, al proceso real corrspondiente
\ \\[2pt]
\section{Diseño del experimento}
\IEEEPARstart{L}a tería de probabilidad proporciona varios modelos matemáticos para la descripción de fenómenos sujeto a influjos csuales, y tiene como objetivo escemcial la compresión matemática de las regularidades de los fenomenos aleatorios.\\
La teoría de probabilidad se construye de manera axiomática, de acuerdo con procedimento probado y muy utilizado hasta hoy día, y se sirve en gran medida de los métodos y resultados del análisis.\\
La Estadística matemática proporciona, sobre la base de la teoría de probabilidades, métodos mediante los cuales se puede obtener información sobre distintas poblaciones e investigar, utilizando datosmuestrales aleatorios;con esto se da origen también a métodos de ajuste de un modelo matemático, que considere efectos aleatorios, al proceso real corrspondiente.\\
\section{Estado del arte}
\IEEEPARstart{L}a tería de probabilidad proporciona varios modelos matemáticos para la descripción de fenómenos sujeto a influjos csuales, y tiene como objetivo escemcial la compresión matemática de las regularidades de los fenomenos aleatorios.\\
La teoría de probabilidad se construye de manera axiomática, de acuerdo con procedimento probado y muy utilizado hasta hoy día, y se sirve en gran medida de los métodos y resultados del análisis.\\
La Estadística matemática proporciona, sobre la base de la teoría de probabilidades, métodos mediante los cuales se puede obtener información sobre distintas poblaciones e investigar, utilizando datosmuestrales aleatorios;con esto se da origen también a métodos de ajuste de un modelo matemático, que considere efectos aleatorios, al proceso real corrspondiente.\\
\begin{itemize}[\IEEEsetlabelwidth{5}]
\item[1] trabajo grupal
\item[2] LALALALALALA
\end{itemize}

\begin{IEEEkeywords}
Tarta de Manzana, Recetas, Manzanas.
\end{IEEEkeywords}
 \end{document}